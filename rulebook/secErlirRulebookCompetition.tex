%!TEX root = ./RoCKIn-D-2.1.6.tex
%--------------------------------------------------------------------
%--------------------------------------------------------------------
%--------------------------------------------------------------------
\newpage
\section{Competition Structure}
\label{sec:CompetitionStructure}

%\revtodo{This section to be written/revised by URome and IST. Add structure as necessary.}


%--------------------------------------------------------------------
%--------------------------------------------------------------------
\subsection{Competition Elements}
\label{ssec:RoawCompetitionElements}

RoCKIn competitions are scientific competitions where the rules are designed in such a way that the rankings also take the role of measurements of the performance of participants, according to objective criteria. This is called in RoCKIn jargon, a \emph{benchmarking competition}.

The elements composing a \emph{benchmarking competition} were defined in RoCKIn Deliverable D1.1 (''Specification of General Features of Scenarios and Robots for Benchmarking Through Competitions''). We recover the most relevance here for the rule book document to be self-contained.
%
\begin{definition}[Functionality] 
One of the basic abilities that a \emph{robot system} is required to possess in order to be subjected to a given \emph{experiment}. 
\end{definition}
%
The list of \emph{functionalities} for \roaw is defined in Section~\ref{sec:FunctionalityBenchmarks}.
%
\begin{definition}[Functional Module] 
The (hardware and/or software) components of a \emph{robot system} that are involved in providing it with a specific \emph{functionality}.
\end{definition}
%
\begin{definition}[Task] 
An operation or set of operations that a \emph{robot system} is required to perform, with a given (set of) goal(s), in order to participate in a \emph{benchmarking competition}. 
\end{definition}
%
The list of \emph{tasks} for \roaw is defined in Section~\ref{sec:TaskBenchmarks}.
%
\begin{definition}[Benchmarking] 
The process of evaluating the performance of a given \emph{robot system} or one of its \emph{functional modules}, according to a specified \emph{metric}. 
\end{definition}
%
\begin{definition}[Benchmark] 
The union of one or more \emph{benchmarking experiments} and a set of \emph{metrics} according to which the course and the outcome of the experiments -- described by suitable data acquired during the experiments -- will be evaluated.
\end{definition}
%
\begin{definition}[Functionality Benchmark] 
A \emph{benchmark} which aims at evaluating the quality and effectiveness of a specific \emph{functional module} of a \emph{robot system} in the context of one or more \emph{scenarios}. 
\end{definition}
%
\begin{definition}[Task Benchmark] 
A \emph{benchmark} which aims at evaluating the quality of the overall execution of a \emph{task} by a \emph{robot system} in the context of a single \emph{scenario}. 
\end{definition}
%
\begin{definition}[Score] 
The result obtained when a \emph{robot system} is subjected to a \emph{benchmark} (\emph{task} \emph{benchmark} or \emph{functionality} \emph{benchmark}). 
\end{definition}
%
The \emph{scores} will be used in the \roaw competitions to order the teams according to their performance in \emph{tasks} and \emph{functionalities}. 

Benchmarking data will be logged by the Organizing Committee for offline, ex-post analysis of team performances in the \roaw \emph{tasks} and \emph{functionalities}, so as to provide relevant scientific information, such as the impact of \emph{functional modules} performance in the \emph{robot system} \emph{task} performance, or to improve the scoring system of future \roaw competitions.

%--------------------------------------------------------------------
%--------------------------------------------------------------------
\subsection{Structure of the Competition}
\label{ssec:RoawCompStructure}

Each \emph{task benchmark} and \emph{functional benchmark} will be performed by each of the competing teams several times, to ensure some level of repeatability of the results.

\emph{Task benchmarks} and \emph{functional benchmarks} will be executed as much as possible in parallel, i.e., while one team executes a \emph{task benchmark}, another team executes a \emph{functional benchmark} simultaneously in another area of the arena.


%--------------------------------------------------------------------
% EOF
%--------------------------------------------------------------------
