%!TEX root = ./ERL Industrial Robots.tex

%--------------------------------------------------------------------
%--------------------------------------------------------------------
%--------------------------------------------------------------------
\section{\erlir Organization}
\label{sec:RoawOrganization}

%--------------------------------------------------------------------
%--------------------------------------------------------------------
\subsection{\erlir Management}
\label{ssec:RoawManagement}

The management structure of \erlir has been divided into three committees, namely \emph{Executive Committee}, \emph{Technical Committee} and the \emph{Organization Committee}. 
Pedro Lima (acting as the overall Coordinator of the challenges execution, within his role as \erl coordinator) is acting as \textbf{Supra-Chair}.
The roles and responsibilities of those committees are described in the following paragraphs.

%--------------------------------------------------------------------
\subsubsection{\erlir Executive Committee}
\label{sssec:RoawEC}

The Executive Committee (EC) is represented by the coordinators of each \erlir partner related to the respective activity area. 
The committee is mainly responsible for the overall coordination of \erlir activities and especially for dissemination in the scientific community. 
%	
\begin{itemize}	\topsep-12pt\itemsep-2pt\parsep0pt
\item Pedro Lima (Instituto Superior T\'ecnico, Portugal)
\item Daniele Nardi (Sapienza Universit\`a di Roma, Italy)
\item Gerhard Kraetzschmar (Bonn-Rhein-Sieg University, Germany)
\item Rainer Bischoff (KUKA Roboter GmbH, Germany)
\item Matteo Matteucci (Politecnico di Milano, Italy)
\end{itemize}


%--------------------------------------------------------------------
\subsubsection{\erlir Technical Committee}
\label{sssec:RoawTC}

The Technical Committee (TC) is responsible for the rules of the league. Each member of the committee is involved in maintaining and improving the current rule set and in the adherence of these rules. Other responsibilities include the qualification of teams, handling general technical issues within the league, deciding about giving awards in case the number of participants is lower than the thresholds specified in Section~\ref{sec:AwardCategories}, as well as resolving any conflicts inside the league during an ongoing competition. The members of the committee are further responsible for maintaining the \erlir infrastructure. 

The Technical Committee currently consists of the following members:
%
\begin{itemize}	\topsep-12pt\itemsep-2pt\parsep0pt
\item Alberto Pretto (Sapienza Universit\`a di Roma, Italy)
\item Rhama Dwiputra (Bonn-Rhein-Sieg University, Germany)
\item Tim Friedrich (KUKA Roboter GmbH, Germany)
\item Matteo Matteucci (Politecnico di Milano, Italy)
\end{itemize}
%
This committee can include members of the Executive Committee.

%--------------------------------------------------------------------
\subsubsection{\erlir Organizing Committee}
\label{sssec:RoawOC}

The Organizing Committee (OC) is responsible for the actual implementation of the competition, i.e.~providing everything what is required to perform the various tests. 
Specifically, this means providing setting up the test arena(s), providing any kind of objects (e.g.~manipulation objects), scheduling the tests, assigning and instructing referees, recording and publishing (intermediate) competition results and any other kind of management and advertisement duties before, during and after the competition. 

The Organizing Committee currently consists of the following members:
%
\begin{itemize}	\topsep-12pt\itemsep-2pt\parsep0pt
\item \textbf{Chair:} Tim Friedrich (KUKA Roboter GmbH, Germany)
\item Francesco Amigoni (Politecnico di Milano, Italy)
\item Tiago Veiga (Instituto Superior T\'ecnico, Portugal)
\item Frederik Hegger (Bonn-Rhein-Sieg University, Germany)
\item Graham Buchanan (InnoCentive EMEA, U.K.)
\end{itemize}
%

%--------------------------------------------------------------------
%--------------------------------------------------------------------
\subsection{\erlir Infrastructure}
\label{ssec:RoawInfrastrcuture}

%--------------------------------------------------------------------
\subsubsection{\erlir Web Page}
\label{sssec:RoawWeb}

The official \erlir website can be reached at
%
\begin{flushleft}
	\hspace*{1cm}\url{http://erl.rocks/} 
\end{flushleft}
%
On this web page, teams can find introductory information about the league itself as well as relevant information about upcoming events, the most recent version of the rule book, videos and pictures of past competitions and links to further resources like the official mailing list or wiki. 


%--------------------------------------------------------------------
\subsubsection{\erlir Mailing List}
\label{sssec:RoawMailingList}

The official \erlir mailing list maintained by the league is as follows
%
\begin{flushleft}
	\hspace*{1cm}\url{laber} 
\end{flushleft}
%
Anyone can subscribe by using the following subscription page.
%
\begin{flushleft}
	\hspace*{1cm}\url{bla}\\ 
\end{flushleft}
%
Every subscriber is requested to register either with an email address which already encodes the real name or alternatively specify it in the provided field at the subscription page. In order to prevent the mailing list from spammers, this mailing list is moderated. 

The mailing list will be used for any kind of official announcement, e.g.~upcoming \erlir competitions, rule changes, registration deadlines or infrastructure changes. Teams are welcome to raise any kind of question regarding the league on this list.

\subsection{\erlir Competition Organization}
\label{ssec:CompOrg}

%--------------------------------------------------------------------
\subsubsection{Qualification and Registration}
\label{sssec:CompQualReg}

Participation in \erlir requires successfully passing a qualification procedure. This procedure is to ensure a well-organized competition event and the safety of participants.
Depending on constraints imposed by a particular site or the number of teams interested to participate, it may not be possible to admit all interested teams to the competition.

All teams that intend to participate at the competition have to perform the following steps (using the forms at the web site \url{http://erl.rocks/}):
%
\begin{enumerate}	\topsep-12pt\itemsep-2pt\parsep0pt
	\item Preregistration (deadline: 31 May 2015) -- optional
	\item Submission of qualification material (e.g.~team description paper and video; deadline: 31 August 2015) -- mandatory
	\item Final registration (between 9 and 30 September 2015) -- mandatory, and for qualified teams only
\end{enumerate} 

%--------------------------------------------------------------------
\paragraph{Preregistration}
\label{par:CompPreregistration}

A team must provide the following information during the preregistration process:
%
\begin{itemize}	\topsep-12pt\itemsep-2pt\parsep0pt
	\item Team name + Affiliation
	\item Team leader name
	\item Team leader email address
	\item Expected number of team members
	\item Whether the team plan to bring their own robot or not
	\item Middleware used for software development
\end{itemize}
%
This step can be considered as an \emph{Intention of Participation} declaration and serves as planning basis for the Organizing Committee.

%--------------------------------------------------------------------
\paragraph{Qualification}
\label{par:CompQualification}

The qualification process serves a dual purpose: It should allow the Technical Committee to assess the safety of the robots a team intents to bring to a competition, and it should allow to rank teams according to a set of evaluation criteria in order to select the most promising teams for a competition, if not all interested teams can be permitted. The TC will select the qualified teams according to the qualification material provided by the teams. 

The evaluation criteria will include:
%
\begin{itemize}	\topsep-12pt\itemsep-2pt\parsep0pt
  \item Team description paper
  \item Team web site
  \item Relevant scientific contribution/publications
  \item Professional quality of robot and software
  \item Novelty of approach
  \item Relevance to industrial service robotics
  \item Performance in other competitions
  \item Contribution to \erlir league (e.g.~by organization of events or provision and sharing of knowledge)
\end{itemize}
%
The Team Description Paper (TDP) is a central element of the qualification process and has to be provided by each team as part of the qualification process. 
The TDP should at least contain the following information in the author/title section of the paper:
%
\begin{itemize}	\topsep-12pt\itemsep-2pt\parsep0pt
  \item Name of the team (title)
  \item Team members (authors), including the team leader
  \item Link to the team web site
  \item Contact information
\end{itemize}
%
The body of the TDP should contain information on the following:
focus of research/research interests:
%
\begin{itemize}	\topsep-12pt\itemsep-2pt\parsep0pt
  \item Description of the hardware, including an image of the robot(s)
  \item Description of the software, especially the functional and software architectures
  \item Main involved research areas in the team work
  \item Innovative technology (if any)
  \item Reusability of the system or parts thereof
  \item Applicability and relevance to industrial robotics
\end{itemize}
%
The team description paper should cover in detail the technical and scientific approach, while the team web site should be designed for a broader audience. Both the web site and the TDP have to be written in English.

The length of the team description paper is limited to 6 pages and has to be to submitted in the IEEE Conference Proceedings format\footnote{\url{http://www.ieee.org/conferences_events/conferences/publishing/templates.html}}.


%--------------------------------------------------------------------
\paragraph{Registration}
\label{par:CompRegistration}

Only if a team has passed the qualification procedure successfully it is allowed to register officially for the competition and has to provide additional information e.g.~the exact number of team members.
Please be advised that this year, a team participating in TBM(s) for one of the Challenges (\roah or \roaw) must participate in all FBMs for that Challenge.
Further information about the registration procedure will be communicated through the mailing list of qualified teams.
The number of people to register per team may be unlimited, but during the competition the organizers will provide space only for 6 persons to work at tables in the team area. 
During the final registration, each team has to designate one member as team leader.  A change of the team leader must be communicated to the Organizing Committee. 

%--------------------------------------------------------------------
\subsubsection{Setup and Schedule}
\label{sssec:CompSetupSchedule}

\erlir competition will take place in the main science museum Pavilion of Knowledge and Portugal Pavilion of Lisboa, from 17-23 November 2015.
%26-30

17--18 November will be the assembly days, during which the arenas, team areas, power, audiovisual equipment and other infrastructure will be put in place.

19--20 November will be setup days, that the teams can use to unpack their robots, calibrate the robot systems, and get information about the test bed, important objects and other relevant details. The site will be closed to the public.

There will be three competition days: 21, 22 and 23 November. During those days, the competitions will occur following the procedures and rules described in the subsections of this document with the same title. The site will be accessible to the public during the actual competitions.

The award and closing ceremony will take place in the evening of the last day, 23 November 2015.

Several satellite events, with the participation of industry and academia stakeholders, will take place during the five days of the main event. These include talks by members of RoCKIn's Advisory Board, and the assessment of the Competition by the members of RoCKIn's Experts Board.


\noindent
\begin{description}
	\item[Schedule:] For the scheduling of particular stages, tests, and technical challenges of the competition the following applies:
	%
	\begin{itemize}
	  	\item The exact schedule of task-functionality tests will be announced one week before the actual competition by the OC on both the website and the mailing list of qualified teams.
	  	\item In order to avoid excess of "traffic" inside the testbed, an additional schedule only for test slots will be established on site by the OC.
	  	\item A set of test slots will be assigned to each team between the official test slots, where a team has exclusive access to the testbed without any other team/robot inside the arena.
	\end{itemize}
	\item[Setup:] For the arrival, setup, and preparation of teams participating in the competition, the following procedures apply: 
	%	
	\begin{itemize}
	  	\item A first draft of the rule book will be made public on June 30th 2015.
		\item Revisions will be possible and updated in the online versions of the document, based on suggestions of all relevant stakeholders (including pre-registered and registered teams) until July 31th 2015.
		\item The final version of the rule book will be made public, no later than eight weeks before the actual event, by the TC, including all the items referred as open in this document (e.g., some benchmarking and scoring items) and revisions resulting from the discussion referred in the previous item.
	  	\item The competition side will be divided into a competition arena and a team area.
	  	\item The competition arena consists of one or more testbeds (the arena) and is open for public.
	  	\item The arena must be kept clean and in a presentable condition all the time.
	  	\item The team area is a dedicated area only for team members, no public access here.
	  	\item Each team will be assigned to a designated area with tables and chairs (based on the number of team members), with power sockets, a shelf internet connection and a reasonable area to park their robot and other equipment.
	\end{itemize}  
\end{description}

%--------------------------------------------------------------------
\subsubsection{Competition Execution}
\label{sssec:CompExec}

\begin{itemize}
\item Referees will be determined by the OC out of the group of team leaders and TC members.
\item The referees ensure the correct execution of a benchmark, are in charge of keeping the time and counting the scores, being always helped by a TC or OC member.
\item In case of any dangerous situation the referees are allowed to immediately stop a benchmark and trigger the emergency stop functionality of the respective robot.
\item The official language for all kind of communication within the league is English (e.g.,~team leader meetings, announcements, schedule).
\item The order in which the teams have to perform a particular benchmark will be determined by a draw through the OC.
\item The order will be announced on the day \underline{before} the actual benchmark.
\item No team members or other persons are allowed to be in the arena during an official benchmark (only if the rule book explicitly allows/requires this).
\item Regular team leader meetings (every day) will be organized and announced by the TC/OC during the competition in order to discuss open issues for upcoming benchmarks.
\end{itemize}

%--------------------------------------------------------------------
\subsubsection{Measurements Recording}
\label{par:CompMeasurements}

Several variables of interest will be recorded by the EC, TC and OC during the actual benchmarks of the teams during the competition, while performing their \emph{task} and \emph{functionality benchmarks}. Some of these will be performed by \erlir equipment, though requiring the installation of markers on the team robots. Other logging will require the teams to accommodate, in their software, modules that respond to solicitations from test bed-installed software. 
Details on these procedures will be provided closer to the competition dates, but the teams must be ready to commit to such requirements as one of the key requirements to be selected for the \erlir competitions.
%--------------------------------------------------------------------
% EOF
%--------------------------------------------------------------------
