%!TEX root = ./ERL Industrial Robots.tex

\newcommand{\myversion}{1}

%%% fix language and font encoding
\usepackage[english]{babel} 
\usepackage[utf8]{inputenc}
\usepackage[T1]{fontenc}

%%% use general utility packages that ease use of other packages
\usepackage{ifthen}
\usepackage{calc}
\usepackage{pbox}

%%% fix the geometry of the document
\usepackage[left=20mm,right=30mm,top=25mm,bottom=25mm]{geometry}

%%% fix setting for margin notes
\setlength{\marginparsep}{2mm}
\setlength{\marginparwidth}{26mm}
\setlength{\marginparpush}{1mm}

%%% adapt percentage of space usable for figures on top and bottom
%%% (default values result in too many figures ending on pages of floats
\renewcommand{\topfraction}{0.9}
\renewcommand{\bottomfraction}{0.9}

%%% fix headings and footings
\newcommand{\erlcopyright}{\copyright\ 2020 by ERL Team}
\usepackage{fancyhdr}
\fancyhf{}
\fancyhead[LO,RE]{\footnotesize\nouppercase{\leftmark}}
\fancyhead[LE,RO]{\footnotesize\nouppercase{\rightmark}}
\fancyhead[C]{\footnotesize}
\fancyfoot[LO,RE]{\footnotesize Revision \revisionNumber}
\fancyfoot[LE,RO]{\fancyplain{\footnotesize\bricscopyright}{\footnotesize\erlcopyright}}
\fancyfoot[C]{\thepage}
\renewcommand{\plainheadrulewidth}{0.6pt}
\renewcommand{\plainfootrulewidth}{0.6pt}
\renewcommand{\headrulewidth}{0.6pt}
\renewcommand{\footrulewidth}{0.6pt}
\pagestyle{fancy}

%%% packages for handling insertion of figures
\usepackage[pdftex]{graphicx}
\usepackage{subfigure}
\usepackage{epstopdf} 
\DeclareGraphicsExtensions{.pdf,.png,.jpg,.gif,.eps,.mps} 

%%% packge helpful for typesetting landscape tables and figures
%%% note: this will rotate the complete figure including captions
%%% note: for rotating inserted picture only use "angle" option in graphicx
\usepackage{rotating}
\usepackage{tablefootnote}

%%% helpful package for multiple author editing and proofreading
\usepackage{todonotes} %[disable]

%%% a package for some special symbols (celsius, ohm, etc.)
\usepackage{textcomp, gensymb}

%%% packages for math environment 
\usepackage{amssymb}
\usepackage{amstext}
%\usepackage{amsopn}
\usepackage{amsthm}
\usepackage{thmtools}

%%% not clear what the next is good for
% It fixes the problem of "undefined control sequence \IfStrEq"
\usepackage{xstring}

%%% the xspace package is useful for handling spacing in macros
\usepackage{xspace}
\usepackage{acronym}

%%% various other packages
\usepackage{multirow}
\usepackage{color}
\usepackage{url}
\usepackage[unicode]{hyperref}

%\usepackage[citestyle=numeric]{biblatex}
%\renewcommand{\bibname}{References}

%setting for better visibility of todonotes - Jakob
\setlength{\marginparwidth}{2cm}

%we will have long table to show alle the objects in the environment
\usepackage{longtable}

%--------------------------------------------------------------------
\declaretheorem[thmbox=M,numberwithin=section,name=Definition]%
{definition}
\declaretheorem[thmbox=M,numberwithin=section,name=Scenario Specification]%
{scenarioSpec}
\declaretheorem[thmbox=M,numberwithin=section,name=Scenario Constraint]%
{scenarioConstraint}
\declaretheorem[thmbox=M,numberwithin=section,name=Environment Specification]%
{envSpec}
\declaretheorem[thmbox=M,numberwithin=section,name=CFH Specification]%
{CFHSpec}
\declaretheorem[thmbox=M,numberwithin=section,name=Environment Constraint]%
{envConstraint}
\declaretheorem[thmbox=M,numberwithin=section,name=Feature Specification]%
{featureSpec}
\declaretheorem[thmbox=M,numberwithin=section,name=Feature Variation]%
{featureVar}
\declaretheorem[thmbox=M,numberwithin=section,name=Input Specification]%
{inputSpec}
\declaretheorem[thmbox=M,numberwithin=section,name=Object Specification]%
{objectSpec}
\declaretheorem[thmbox=M,numberwithin=section,name=Object Constraint]%
{objectConstraint}
\declaretheorem[thmbox=M,numberwithin=section,name=Robot Specification]%
{robotSpec}
\declaretheorem[thmbox=M,numberwithin=section,name=Robot Constraint]%
{robotConstraint}
\declaretheorem[thmbox=M,numberwithin=section,name=Task Specification]%
{taskSpec}
\declaretheorem[thmbox=M,numberwithin=section,name=Task Constraint]%
{taskConstraint}
\declaretheorem[thmbox=M,numberwithin=section,name=Rule]%
{rockinRule}


%----project defs------------------------------------------------
\newcommand{\ro}{RoCKIn\xspace}
\newcommand{\rollin}{RoCKIn\-'N'\-RoLLIn\xspace}
\newcommand{\roaw}{RoCKIn\-@Work\xspace}
\newcommand{\roah}{RoCKIn\-@Home\xspace}

\newcommand{\rc}{Robo\-Cup\xspace}
\newcommand{\rcaw}{Robo\-Cup\-@Work\xspace}
\newcommand{\rcah}{Robo\-Cup\-@Home\xspace}

\newcommand{\rockeutwo}{RockEU2\xspace}
\newcommand{\erl}{European Robotics League\xspace}
\newcommand{\erlir}{ERL-PSR\xspace}
\newcommand{\erlsr}{ERL-CSR\xspace}
\newcommand{\erlirlong}{ERL Professional Service Robots\xspace}
\newcommand{\erlsrlong}{ERL Consumer Service Robots\xspace}

\newcommand{\wifi}{WLAN}
\newcommand{\wlan}{WLAN}

\newcommand{\CFH}{Central Factory Hub}

\newcommand{\myNutshellAlone}{no}
\newcommand{\myRuleBookAlone}{no}
\newcommand{\myCompetitionDesignAlone}{no}

%\DeclareMathOperator*{\argmin}{arg\,min}
%\DeclareMathOperator*{\argmax}{arg\,max}
%\newcommand{\x}{\mathbf{x}}
%\newcommand{\X}{\mathbf{X}}
%\newcommand{\p}{\mathbf{P}}
%\newcommand{\z}{\mathbf{z}}
%\newcommand{\e}{\mathbf{e}}
%\newcommand{\la}{\mathbf{l}}
%\newcommand{\ob}{\mathbf{o}}
%\newcommand{\Ob}{\mathbf{O}}
%\newcommand{\w}{\mathbf{w}}
%\newcommand{\Om}{\mathbf{\Omega}}
%\newcommand{\Sig}{\mathbf{\Sigma}}
%\newcommand{\vo}{\mathbf{v}}
%\newcommand{\ur}{\mathbf{u}}
%\newcommand{\zl}{\mathbf{z}}
%\newcommand{\Xp}{\mathcal{X}}
%\newcommand{\Px}{\mathbf{P}}
%\newcommand{\Ze}{\mathbf{0}}
%\newcommand{\rob}{\mathcal{L}}
%\newcommand{\deriv}[1]{\frac{\partial}{\partial #1}}

%----review defs-------------------------------------------------
%\usepackage{trackchanges}
\usepackage[normalem]{ulem}

\definecolor{myblack}{rgb}{0,0,0}
\definecolor{mygrey}{rgb}{0.8,0.8,0.8}
\definecolor{myred}{rgb}{1.0,0,0}
\definecolor{mygreen}{rgb}{0,1.0,0}
\definecolor{myblue}{rgb}{0,0,1.0}
\definecolor{mywhite}{rgb}{1.0,1.0,1.0}

\newcommand{\revadd}[1]{\textcolor{blue}{#1}}
\newcommand{\revdel}[1]{\textcolor{red}{\sout{#1}}}
\newcommand{\revcom}[1]{\textcolor{magenta}{(#1)}}
\newcommand{\revcha}[2]{\revdel{#1}\revadd{#2}}

%\newcommand{\revauthor}{GKK}
%\newcommand{\revauthor}{NH}
%\newcommand{\revauthor}{IA}
%\newcommand{\revauthor}{RD}
%\newcommand{\revauthor}{FH}
%\newcommand{\revauthor}{SN}
%\newcommand{\revauthor}{MM}
%\newcommand{\revauthor}{PL}
%\newcommand{\revauthor}{AA}
%\newcommand{\revauthor}{DN}
%\newcommand{\revauthor}{LI}
\newcommand{\revauthor}{JBr}
%\newcommand{\revauthor}{RB}

%\newcommand{\revauthor}{BRSU} % blue
%\newcommand{\revauthor}{POLI} % green
%\newcommand{\revauthor}{IST}  % red
%\newcommand{\revauthor}{ROMA} % magenta
%\newcommand{\revauthor}{KUKA} % orange
%\newcommand{\revauthor}{INNO} % yellow

\newcommand{\revisor}[2]{%
	\renewcommand{\revcolor}{#1}
	\renewcommand{\revauthor}{#2}
	}

\newlength{\revcomwidth}
\setlength{\revcomwidth}{\textwidth}
\addtolength{\revcomwidth}{-20mm}

\newcommand{\revcolor}{myblack}

\newcommand{\revbox}{\textcolor{\revcolor}{$\Box$}\xspace}

%\newcommand{\revdel}[1]{\textcolor{\revcolor}{\sout{#1}}}
%\newcommand{\revadd}[1]{\textcolor{\revcolor}{#1}}
\newcommand{\revmod}[2]{\textcolor{\revcolor}{\sout{#1} #2}}

\newcommand{\revmar}{%
	\marginpar{\textcolor{\revcolor}{\footnotesize \revauthor:}
}}
\newcommand{\revcommar}[1]{%
	\marginpar{\textcolor{\revcolor}{\footnotesize \revauthor: #1}
}}
\newcommand{\revcomtext}[1]{%
	\revbox \textcolor{\revcolor}{#1} \revbox\xspace
}

%\newcommand{\revcom}[1]{%
%	\revbox \textcolor{\revcolor}{#1} \revbox\revcommar{ToDo}\xspace
%}
\newcommand{\revtodo}[1]{%
	\todo[inline,color=\revcolor!33]{\revbox #1 \revbox}%
}
\newcommand{\revtodomar}[1]{%
	\todo[inline,color=\revcolor!33]{\revbox #1 \revbox}%
	\revcommar{}
}
\newcommand{\revTBC}{\revcomtext{To be confirmed.}}
\newcommand{\revTBD}{\revcomtext{To be determined.}}
\newcommand{\revNR}{\revcomtext{Not relevant.}}
\newcommand{\revNA}{\revcomtext{Not applicable.}}
\newcommand{\revTBCfill}{\hfill\revcomtext{To be confirmed.}}
\newcommand{\revTBDfill}{\hfill\revcomtext{To be determined.}}
\newcommand{\revNRfill}{\hfill\revcomtext{Not relevant.}}
\newcommand{\revNAfill}{\hfill\revcomtext{Not applicable.}}

%--------------------------------------------------------------------
\newcommand{\mynote}[1]{{\colorbox{mygrey}{\textbf{#1}}}}
	
\newcommand{\myC}{\textsuperscript{\textcopyright}}
\newcommand{\myR}{\textsuperscript{\textregistered}}
\newcommand{\myTM}{\textsuperscript{\texttrademark}}
	
\newcommand{\mySetTheoremname}[1]{%
	\renewcommand{\listtheoremname}{#1}
	\addcontentsline{toc}{chapter}{#1}
}
	
%--------------------------------------------------------------------
%%% In the following line please replace 'IST_A_CMYK_POS.eps' by 
%%% any of the following options to select the 
%%% logo of the institution that will appear in
%%% the title page
%%% option 1: IST_A_CMYK_POS.eps
%%% option 2: brsu_logo.eps
%%% option 3: Sapienza_Universit_di_Roma.eps
%%% option 4: Politecnico_di_Milano.eps
%%% option 5: top_kuka.png
%%% option 6: innoentivelogo.pdf
\newcommand{\partnerNameForLogo}{consortiumLogo.eps}

%%% In the following line please remove the expression 
%%% ``The Document Title Goes Here'' by the 
%%% document or report title you want to have
\newcommand{\documentTitle}{Competition Design, Rule Book, \\
	and Scenario Construction for \roaw\\}

%%% In the following line please remove the expression 
%%% ``The name(s) of the author(s) go here'' by the  
%%% author(s) name(s).
\newcommand{\documentAuthors}{%
Coordinating Editors:\\  
Tim Friedrich, Rhama Dwiputra, Gerhard Kraetzschmar\\
Contributing Authors:\\ 
Rainer Bischoff,\\ Andrea Bonarini, Rhama Dwiputra, 
Frederik Hegger, Nico Hochgeschwender, \\ Luca Iocchi, 
Gerhard Kraetzschmar,
Pedro Lima, Matteo Matteucci, Daniele Nardi,\\ Viola Schaffionati and Sven Schneider\\
}

%%% In the following line please remove the expression 
%%% ``Put name of lead contractor here' by the  
%%% Lead contractor's name.
\newcommand{\leadContractor}{KUKA Roboter GmbH}

%%% In the following line please remove the expression 
%%% ``Put revision number here' by the  
%%% revision number.
\newcommand{\revisionNumber}{1.2}
\newcommand{\revisionNumberNutshell}{1.2}
\newcommand{\revisionNumberRuleBook}{1.2}

%%% Please DO NOT MODIFY the following 6 lines below
\IfStrEq{\partnerNameForLogo}{IST_A_CMYK_POS.eps}%
{\newcommand{\institutionLogoHeight}{1.8cm}}{}
\IfStrEq{\partnerNameForLogo}{brsu_logo.eps}%
{\newcommand{\institutionLogoHeight}{1.1cm}}{}
\IfStrEq{\partnerNameForLogo}{Sapienza_Universit_di_Roma.eps}%
{\newcommand{\institutionLogoHeight}{1.8cm}}{}
\IfStrEq{\partnerNameForLogo}{Politecnico_di_Milano.eps}%
{\newcommand{\institutionLogoHeight}{1.9cm}}{}
\IfStrEq{\partnerNameForLogo}{top_kuka.png}%
{\newcommand{\institutionLogoHeight}{0.75cm}}{}
\IfStrEq{\partnerNameForLogo}{consortiumLogo.eps}%
{\newcommand{\institutionLogoHeight}{3cm}}{}

%--------------------------------------------------------------------
%%% Number of section layers to show in TOC (i.e. section, subsection, subsubsection)
\setcounter{tocdepth}{3} 

%%% Number of subsection layers to allow in document
%\setcounter{secnumdepth}{5} 
\setcounter{secnumdepth}{3} 


%--------------------------------------------------------------------
% EOF
%--------------------------------------------------------------------
