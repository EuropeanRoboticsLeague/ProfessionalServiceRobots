%!TEX root = ./ERL Industrial Robots.tex

\begin{abstract}
The objective of the \erl is to organize several indoor robot competition events per year, ensuring a
scientific competition format, around the following two challenges: \erlsr and \erlir.\\
Those indoor robot competitions will be focused on two major challenges addressed by H2020: societal
challenges (service robots helping and interacting with humans at home, especially the elderly and
those with motor disabilities) and industrial leadership (industrial robots addressing the flexible factories of the future and modern automation issues). These challenges were addressed by \roah
and \roaw and will be extended in \rockeutwo by building on the current version of the
rule books and testbeds designed and used during RoCKIn’s project lifetime.\\
Benchmarking and scoring procedures will be defined taking into account the specifications of the \erlsr and \erlir challenges. The novel \erl (ERL) competitions format aims to become a sustainable distributed format (i.e. not a single big event) which is similar to the format of the European Football Champions League, where the role of national leagues is played by existing test beds (e.g., the RoCKIn test beds, but also the ECHORD++ RIFs), used as meeting points for “matches” where one or more teams visit the home team for a small tournament. This format will exploit arenas temporarily available during major competitions in Europe allowing the realization of larger events with more teams.\\
For local tournaments teams will be encouraged to arrive 1-2 weeks before the actual competition to participate in an integration week, where the hosting institution provides technical support on using the local infrastructure. This will ensure a higher team technical readiness level (TTRL), which concerns the ability of a team to have its robot(s) running without major problems, using modular software that ensures quick adaptation and composition of functionalities into tasks, and to use flawlessly the competition infrastructure, whose details may change from event to event.\\
\rockeutwo will provide a certification process to assess any new candidates testbeds as RIFs for both challenges, based on the
\erl rule book specifications and the implementation of the proper benchmarking and scoring procedures. This will enable the creation of a network of European robotics testbeds having the specific purpose of benchmarking domestic robots, innovative industrial robotics applications and Factory of the Future scenarios.\\
\erlir rules will promote research on robust execution of industrially-relevant tasks for mobile
manipulators and their seamless integration in Factory of the Future flexible automation scenarios. This
necessitates the participating teams, research groups, and companies to pursue research in all aspects
required to provide the relevant functionalities, including robust perception of self-similar, textureless
objects, robust state estimation and situation assessment techniques, and efficient navigation and
manipulation capabilities. Special attention is paid to enhance the rules towards scenarios involving
multiple robots performing tasks cooperatively or competitively while operating in a single shared
environment, more dynamic integration with overall factory control (needed for FoF integration), and
possibly first scenarios involving assistance to human workers. The \erl competitions and
benchmarking exercises provide coherent methods that allow to objectively assess and compare
performance and technology readiness levels.
\end{abstract}

%--------------------------------------------------------------------
% EOF
%--------------------------------------------------------------------
