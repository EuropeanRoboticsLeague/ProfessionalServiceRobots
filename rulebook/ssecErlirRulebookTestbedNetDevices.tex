%!TEX root = ./ERL Industrial Robots.tex


%--------------------------------------------------------------------
%--------------------------------------------------------------------
\subsection{Networked Devices in the Environment}
\label{ssec:NetworkedDevices}
There are various networked devices available in the \erlir environment. 
The following description provides an overview on the capability of each networked device.
An interface description for each networked device is provided in detail in the task benchmark section (see Section \ref{sec:TaskBenchmarks}).
Both networked devices are used for delivering parts to the \erlir arena and they are operated by the referee(s).


\begin{figure}[htb]
	\begin{center}
%		\hfill
%		\subfigure[Drilling machine]{
%			\scalebox{1.0}[1.0]{
%				\includegraphics[height=28mm,angle=90,trim=0px 0px 0px 0px,clip]%
%				{pics/atwork/networked_devices/drillingMachine.jpg}%
%			}
%			\label{fig:nutshellcoverPlateDrillingMachine}
%		}%
		\hfill
		\subfigure[Conveyor belt]{
			\scalebox{1.0}[1.0]{%
				\includegraphics[height=28mm,angle=90,trim=0px 0px 0px 0px,clip]%
				{pics/atwork/networked_devices/QCC.jpg}%
			}%
			\label{fig:nutshellcoverPlateQCC}
		}%
%		\hfill
%		\subfigure[Force fitting machine]{
%			\scalebox{1.0}[1.0]{%
%				\includegraphics[height=28mm,angle=90,trim=0px 0px 0px 0px,clip]%
%				{pics/atwork/networked_devices/forceFittingMachine.jpg}%
%			}%
%			\label{fig:nutshellForceFittingMachine}
%		}%
		\hfill
		\subfigure[Rotating table]{
			\scalebox{1.0}[1.0]{%
				\includegraphics[height=28mm,angle=0,trim=0px 0px 0px 0px,clip]%
				{pics/atwork/networked_devices/rotatingTable.jpg}%
			}%
			\label{fig:nutshellForceFittingMachine}
		}%
		\hfill\mbox{}
		\caption{Networked devices within the \erlir environment.}
		\label{fig:NutshellNetworkedDevices} 
	\end{center}
\end{figure}

%	\item \emph{Force fitting machine.}
%	The force fitting machine is used for the insertion of a bearing into a bearing box.
%	The force fitting process is performed by first inserting a bearing box with bearing on top of the bearing box. 
%	The placement process is executed with the help of an assembly aid tray.
%	After the bearing box and bearing is properly placed, the force fitting machine is instructed to move down. 
%	Finally, the force fitting machine is instructed to move up again and the processed bearing box and bearing can be picked up (Figure \ref{fig:RoawNetForceFit}).
%	The force fitting machine is used in the \emph{prepare assembly aid tray for force fitting} task.
%\begin{figure}[htb]
%  \begin{center}
%	  \hfill
%		 \subfigure[]{
%		  \scalebox{1.0}[1.0]{
%  		  \includegraphics[height=35mm,angle=90,trim=450px 50px 0px 0px,clip]
%	  		{fig/forceFittingMachine.jpg}
%			}
%			\label{fig:RoawNetForceFit2}
%		}
%  	  \hfill
%	  \subfigure[]{
%		  \scalebox{1.0}[1.0]{
%  		  \includegraphics[height=40mm,angle=0,trim=0px -250px 0px 0px,clip]
%	  		{fig/roawTBM1.jpg}
%			}
%		   \label{fig:RoawNetForceFit1}
%		}
%		\hfill\mbox{}
%	  \caption{The \erlir force fitting machine (a) and bearing boxes in the middle of force fitting process (b). The first bearing (right side) has been successfully force fitted into a bearing box.}
%  	\label{fig:RoawNetForceFit} 
%	\end{center}
%\end{figure}

%	\item \emph{Drilling machine.}
%	The drilling machine is used for drilling a cone sink in a cover plate.
%	The drilling machine is equipped with a customized fixture for the plates.
%	Similar to the force fitting machine, the drilling machine is operated by first inserting the cover plate into the fixture of the drilling machine. 
%	The cover plate placement is followed by moving the drill head down.
%	Finally, the drill is moved up again and the drilled cover plate can be picked up.
%	The drilling machine is used in the plate drilling task specifically in the correction of a faulty cover plate.
%\begin{figure}[htb]
%  \begin{center}
%	  \hfill
%		 \subfigure[]{
%		  \scalebox{1.0}[1.0]{
%  		  \includegraphics[height=35mm,angle=90,trim=0px 0px 0px 0px,clip]
%	  		{fig/drillingMachine.jpg}
%			}
%			\label{fig:RoawNetDrill1}
%		}
%  	  \hfill
%	  \subfigure[]{
%		  \scalebox{1.0}[1.0]{
%  		  \includegraphics[height=52mm,angle=0,trim=900px 150px 150px 250px,clip]
%	  		{fig/roawTBM2.jpg}
%			}
%		   \label{fig:RoawNetDrill2}
%		}
%		\hfill\mbox{}
%	  \caption{The \erlir drilling machine (a) and a cover plate which is placed in the fixture of the drilling machine (b).}
%  	\label{fig:RoawNetDrill} 
%	\end{center}
%\end{figure}




%\item \emph{Quality control camera.} 
%The quality control camera or QCC is placed on top of the conveyor belt and it is used to acquire information on the quality of incoming cover plates delivery through the conveyor belt. 
%The QCC has the responsibility to deliver one cover plate through the conveyor belt (until the cover plate reaches the exit ramp of the conveyor belt) for each received command.
%After receiving a command, the QCC operates the conveyor belt until a cover plate is within the QCC range of view where the QCC will detect any defects on the cover plate.
%The conveyor belt will keep moving until it is stopped by QCC when the cover plate is at the exit ramp of the conveyor belt.
%The QCC it is used for the plate drilling task.


%--------------------------------------------------------------------
% EOF
%--------------------------------------------------------------------
