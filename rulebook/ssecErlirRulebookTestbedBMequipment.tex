%!TEX root = ./ERL Industrial Robots.tex

%--------------------------------------------------------------------
%--------------------------------------------------------------------
\subsection{Benchmarking Equipment in the Environment}
\label{ssec:BenchmarkingEquipment}

\erlir benchmarking is based on the processing of data collected in two ways:
\begin{itemize}
\item \textbf{internal benchmarking data}, collected by the robot system 
	under test (see Section \ref{sec:RobotsTeams});
\item \textbf{external benchmarking data}, collected by the equipment 
	embedded into the testbed.
\end{itemize}
External benchmarking data is generated by the \erlir testbed with a multitude of methods, depending on their nature.

One of the types of external benchmarking data used by \erlir are pose data about robots and/or their constituent parts. To acquire these, \erlir uses a camera-based commercial motion capture system (e.g. NaturalPoint OptiTrack), composed of dedicated hardware and software. Benchmarking data has the form of a time series of poses of rigid elements of the robot (such as the base or the wrist). There is a case where the motion capture system influences the robot perception system. It is the teams' responsibility to check this effect and coordinate with the benchmarking team in order to minimize this effect. Once generated by the OptiTrack system, pose data are acquired and logged by a customized external software system based on ROS (Robot Operating System): more precisely, logged data is saved as \emph{bagfiles} created with the \emph{rosbag} utility provided by ROS. Pose data is especially significant because it is used for multiple benchmarks. There are other types of external benchmarking data that \erlir acquires: however, these are usually collected using devices that are specific to the benchmark. For this reason, such devices are described in the context of the associated benchmark, rather than here. 

Finally, equipment to collect external benchmarking data includes any \emph{server} which is part of the testbed and that the robot subjected to a benchmark has to access as part of the benchmark. Communication between servers and robot is performed via the testbed's own wireless network (see Section \ref{ssec:RobotBenchEquip}).

%--------------------------------------------------------------------
% EOF
%--------------------------------------------------------------------
