%!TEX root = ./ERL Industrial Robots.tex


%--------------------------------------------------------------------
%--------------------------------------------------------------------
\subsection{Manipulation Pick Functionality}
\label{ssec:Manipulation}

%--------------------------------------------------------------------
\subsubsection{Functionality Description}
\label{sssec:ManipulationDescription}

This functionality benchmark assesses the robot's capability of grasping different objects. 
An object from a known set of possible objects is presented in the test for the robot to be identified and grasped.
After identifying the object, the robot needs to perform the grasping motion, lift the object and notify that the object is acquired.

%--------------------------------------------------------------------
\subsubsection{Feature Variation}
\label{sssec:FBMManipulationVariation}

The objects used in the benchmark will be selected from the list of parts to manipulate as presented in Section \ref{ssec:Objects}.
Additionally, the precise position of the object differs in each test.
%--------------------------------------------------------------------
\subsubsection{Input Provided}
\label{sssec:FBMManipulationInput}

The team will be provided with the following information:
\begin{itemize}
	\item The list of possible objects used in the functionality benchmark.
	\item Possible placement of each object used in the functionality benchmark.
\end{itemize}

%--------------------------------------------------------------------
\subsubsection{Expected Robot Behaviour or Output}
\label{sssec:FBMManipulationOutput}

The robot is placed in front of the test area (a planar surface).
One object will be placed in the test area and 
the robot will identify the object and move its end effector on top of it. 
Afterwards, the robot performs the grasping motion of the object and notifies that the grasping has occurred.
The task is repeated with different objects.



%--------------------------------------------------------------------
\subsubsection{Procedures and Rules}
\label{sssec:FBMManipulationProcedures}

The maximum time allowed for one functionality run is 4 minutes (30 seconds for preparation and 210 seconds for execution). A run consists of (1) a preparation phase where the robot is going to its initial configuration and releases a previously grasp object and (2) an execution phase in which the robot detects, localizes, recognizes and manipulates one object.

\begin{description}
\item[Step 1] An object of unknown class and unknown instance will be placed on a table in front of the robot.
\item[Step 2] The robot must determine the object’s class and its instance within that class.
\item[Step 3] The robot must grasp the object, lift it, and notify that grasping has occurred.
\item[Step 4] The robot must keep the grip for a given time while the referee verifies the correct manipulation of the object.
\item[Step 5] The preceding steps are repeated with five different objects.
\end{description} 

For each presented object, the robot must produce the result consisting of:
\begin{itemize}
    \item object class name [string], e.g. containers.
    \item object instance name [string], e.g. EM-02.
\end{itemize}

\subsubsection{Communication with CFH}
\label{sssec:FBMManipulationCommCFH}

For this functionality benchmark the robot does not have to control any networked device in the environment. Only the communication as described below is necessary:

\begin{description}
\item[Step 1] The robot sends a \textbf{BeaconSignal} message at least every second.
\item[Step 2] The robot waits for a \textbf{BenchmarkState} message. It starts the preparation procedure when the \emph{phase} field is equal to PREPARATION and the \emph{state} field is equal to RUNNING.
\item[Step 3] As soon as the robot finishes the preparation phase, it sends a message of type \textbf{BenchmarkFeedback} to the CFH with the \emph{phase\_to\_terminate} field set to PREPARATION. The robot should do this until the \textbf{BenchmarkState}'s \emph{phase} and \emph{state} fields have changed.
\item[Step 4] The robot waits for a \textbf{BenchmarkState} message. It starts the benchmark execution when the \emph{phase} field is equal to EXECUTION and the \emph{state} field is equal to RUNNING.
\item[Step 5] As soon as the robot has finished manipulating the object, it sends a message of type \textbf{BenchmarkFeedback} to the CFH with the required results and the \emph{phase\_to\_terminate} field set to EXECUTION. The robot should do this until the \textbf{BenchmarkState}'s \emph{phase} and \emph{state} fields have changed.
\item[Step 6] The robot continues with Step 2.
\item[Step 7] The functionality benchmark ends when the \textbf{BenchmarkState}'s \emph{phase} field is equal to EXECUTION  and the \emph{state} field is equal to FINISHED.

\end{description}
\noindent
The messages to be sent and to be received can be seen on the Github repository located at \cite{rockin:CFHMessages}.


%--------------------------------------------------------------------
\subsubsection{Acquisition of Benchmarking Data}
\label{sssec:FBMManipulationData}

General information on the acquisition of benchmarking data is described in Section \ref{sec:TbmAcquisitionOfData}. There, the \textbf{offline} part of the benchmarking data can be found.
%
\paragraph{Online data}
In order to send online benchmarking data to the CFH, the robot has to use the \textbf{BenchmarkFeedback} message. The message contains:

\begin{itemize}
	\item grasp\_notification (type: bool)
	\item object\_class\_name (type: string)
	\item object\_instance\_name (type: string)
\end{itemize}

The \textbf{BenchmarkFeedback} message can be found at \cite{rockin:CFHMessages}.

\paragraph{Offline data} 
The additional information described in the following table has to be logged:
\begin{table}[h]
	\centering
	\begin{footnotesize}
		\begin{tabular}{|l|l|l|l|}
			\hline
			Topic				 					&	Type		&	Frame Id		&	Notes \\ \hline\hline
			/rockin/grasping\_pose\tablefootnote{Pose of the grasping position on the object.} 	& geometry\_msgs/PoseStamped & /base\_link & 10 Hz \\ \hline
			/rockin/gripper\_pose\tablefootnote{Pose of the gripper.}	& geometry\_msgs/PoseStamped & /base\_link & 10 Hz \\ \hline
			/rockin/arm\_joints\tablefootnote{Joints data}	& geometry\_msgs/JointState & /base\_link & 10 Hz \\ \hline
		\end{tabular}
	\end{footnotesize}
\end{table}

%--------------------------------------------------------------------
\subsubsection{Scoring and Ranking}
\label{sssec:FBMManipulationScoring}

Evaluation of the performance of a robot according to this functionality benchmark is based on:
%
\begin{enumerate}
\item Number and percentage of correctly identified objects;
\item Number and percentage of correctly grasped objects (the object stops touching the table, see definition below);
\item Execution time (if less than the maximum allowed for the benchmark).
\end{enumerate}
The scoring of teams is based on the number of objects correctly grasped. 
A correct grasp is defined as the object being lifted from the table so to be possible for the judge to pass a hand below it. 
For a grasping to be ``correct'' the position has to be kept for at least 5 seconds from the time the judge has passes the hand below the object. 
The time the judge will require to verify the lifting of the object might be up to 10 seconds. 
In case of ties the overall execution time will be taken into account.

%--------------------------------------------------------------------
% EOF
%--------------------------------------------------------------------
